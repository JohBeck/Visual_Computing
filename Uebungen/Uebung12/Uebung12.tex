\documentclass[accentcolor=tud9c,colorbacktitle,inverttitle,landscape,german,presentation,t]{tudbeamer}
\usepackage{amstext}
\usepackage{amsmath}
\usepackage{graphicx}
\usepackage{multicol}
\usepackage{mathtools}
\usepackage{subfigure}
\usepackage[ngerman,english]{babel}
\usepackage[utf8]{inputenc}
\usepackage{colortbl}
\usepackage{adjustbox}

\begin{document}

\title{\"Ubung 12 - Gruppe 142}
\subtitle{Visual Computing - Farbe}


\author[Johannes Beck, Christian Eilers, Robin Menzenbach, Martin Steinborn]{Johannes Beck, Christian Eilers, Robin Menzenbach, Martin Steinborn}


\date{\today}

\begin{titleframe}
\end{titleframe}

\section{Aufgabe 1} 
	\begin{frame}
		\frametitle{Aufgabe 1: Allgemeine Fragen}
		\begin{itemize}
			\item[a)]%TODO: Erklären Sie den Unterschied zwischen Helligkeit und relativer Helligkeit.
			\item[b)]%TODO: Erklären Sie, was achromatische Farben sind und geben Sie drei Beispiele dazu.
			\item[c)]%TODO: Welche Technischen Farbräume gibt es?
			\item[d)]%TODO: Erklären Sie, bei welchen diskreten Spektren welche Farben wahrgenommen werden und nennen Sie mind. 4 wichtige Farben.
		\end{itemize}
	\end{frame}

\section{Aufgabe 2}
	\begin{frame}
		\frametitle{Aufgabe 2: Farbräume}
		%TODO: Rechnen Sie (17, 36, 28) aus dem 24-Bit-RGB-Farbraum in den HSV-Farbraum um. Runden Sie die Zwischenergebnisse sowie das Endergebnis auf 4 Nachkommastellen. Was drücken die einzelnen Werte jeweils aus?
	\end{frame}

\section{Aufgabe 3}
	\begin{frame}
		\frametitle{Aufgabe 3: Farbwahrnehmungsphänomene}
		%TODO: Im Folgenden sehen Sie zwei Farbwahrnehmungsphänomene. Benennen Sie diese, beschreiben Sie kurz den Effekt, den er beim Menschen auslöst und erklären Sie kurz, wie dieser zustande kommt.
		\begin{itemize}
			\item[a)]
			\item[b)]
		\end{itemize}
	\end{frame}
\end{document}
