\documentclass[accentcolor=tud9c,colorbacktitle,inverttitle,landscape,german,presentation,t]{tudbeamer}
\usepackage{amstext}
\usepackage{amsmath}
\usepackage{graphicx}
\usepackage{multicol}
\usepackage{mathtools}
\usepackage{subfigure}
\usepackage[ngerman,english]{babel}
\usepackage[utf8]{inputenc}
\usepackage{colortbl}
\usepackage{adjustbox}

\begin{document}

\title{\"Ubung 12 - Gruppe 142}
\subtitle{Visual Computing - Farbe}


\author[Johannes Beck, Christian Eilers, Robin Menzenbach, Martin Steinborn]{Johannes Beck, Christian Eilers, Robin Menzenbach, Martin Steinborn}


\date{\today}

\begin{titleframe}
\end{titleframe}

\section{Aufgabe 1} 
	\begin{frame}
		\frametitle{Aufgabe 1: Allgemeine Fragen}
		\begin{itemize}
			\item[a)]% Erklären Sie den Unterschied zwischen Helligkeit und relativer Helligkeit.
				Helligkeit(Brightness) beschreibt das Attribut der Farbwahrnehmung, welches die Menge des abgestrahlten Lichts einer Fläche beschreibt. Relative Helligkeit(Lightness) dagegen beschreibt die Helligkeit einer Fläche im Vergleich zu einer unter gleicher Beleuchtung weiß erscheinenden Fläche.
			\item[b)]% Erklären Sie, was achromatische Farben sind und geben Sie drei Beispiele dazu.
				Unter achromatischen oder unbunten Farben versteht man Farben ohne wahrgenommenen Farbton. Beispiele hierfür sind Schwarz Weiß und Grau.
			\item[c)]% Welche Technischen Farbräume gibt es?
				Geräte RGB, Geräteunabhängige RGB (sRGB, Adobe RGB),YCbCr, HSI(HSV, HSL), CMY / CMYK
			\item[d)]% Erklären Sie, bei welchen diskreten Spektren welche Farben wahrgenommen werden und nennen Sie mind. 4 wichtige Farben.
				Diskrete Spektren weisen für diskrete, getrennte Stellen höhere Intensitäten auf, welche vom Material abhängig sind. Ein Beispiel ist die Quecksilberdampflampe, welche die wichtigen Farben Blau (435nm), Violett (404nm), Grün (546nm) und Orange (577nm) hat.
			
				
		\end{itemize}
	\end{frame}

\section{Aufgabe 2}
	\begin{frame}
		\frametitle{Aufgabe 2: Farbräume}
		% Rechnen Sie (17, 36, 28) aus dem 24-Bit-RGB-Farbraum in den HSV-Farbraum um. Runden Sie die Zwischenergebnisse sowie das Endergebnis auf 4 Nachkommastellen. Was drücken die einzelnen Werte jeweils aus?
			\begin{enumerate}
			\item Zunächst müssen die RGB Werte normiert werden. $ RGB_{norm} =
			\begin{bmatrix}
				R&G&B
			\end{bmatrix}^T \stackrel{!}{\in} [0,1]^3$
			$\Rightarrow RGB_{norm} = \frac{RGB_{24bit}}{255} = \begin{bmatrix}
			0.0667&0.1412&0.1098
			\end{bmatrix}^T$
			\item Anschließend werden der größte und der kleinste Wert ermittelt. $MAX = max(R, G, B) = G = 0.1412; MIN = min(R, G, B) = R = 0.0667$
			\item  $H = 60 ^{\circ} \cdot (2 + \frac{B-R}{MAX-MIN}) = 154.7368^{\circ}$
			\item  H beschreibt den Farbwert(hue) und stellt den Farbwinkel auf dem Farbkreis dar.
			\end{enumerate}
		\end{frame}
		\begin{frame}
			\frametitle{Aufgabe 2: Farbräume (2)}
			\begin{enumerate}
				\item $S = \frac{(MAX-MIN)}{MAX} = 0.5278$
				\item S bechreibt die Farbsättigung(saturation) und gibt einen prozentualen Verlauf an, wobei 0\% ein Neutralgrau bezeichnet und 100\% die gesättigte, reine Farbe darstellt \
				\item $V = MAX = G = 0.1412$
				\item V beschreibt den Hellwert (value) der Farbe. Auch er wird prozentual angegeben, wobei 0\% keiner Helligkeit und 100\% voller Helligkeit entspricht
				\item $HSV = \begin{bmatrix}154.7368^{\circ}, 52.78\%, 14.12\%\end{bmatrix}$
			\end{enumerate}
		\end{frame}

\section{Aufgabe 3}
	\begin{frame}
		\frametitle{Aufgabe 3: Farbwahrnehmungsphänomene}
		% Im Folgenden sehen Sie zwei Farbwahrnehmungsphänomene. Benennen Sie diese, beschreiben Sie kurz den Effekt, den er beim Menschen auslöst und
		\begin{itemize}
			\item[a)]Crispening-Effekt: Durch die Wahl einer ähnlichen Hintergrundfarbe wird der Farbunterschied zweier ähnlicher Farbreize vergrößert. Bei ähnlichen Lichtreizen, kann man die Helligkeitsunterschiede stärker wahrnehmen. Durch den stark hellen und dunklen Hintergrund in den zwei anderen Bildern, ist der Unterschied zwischen Rot und Schwarz/Weiß so stark, dass der geringe Unterschied zwischen den Rottönen nicht klar (crisp) hervortritt.
			\item[b)]Simultankontrast: Die wahrgenommene Farbe eines Farbreizes wird durch den Hintergrund beeinflusst, auf dem er präsentiert wird. So wird eine Farbe auf hellem Hintergrund dunkler wahrgenommen und umgekehrt. Bei der Wahrnehmung einer Farbfläche wird gleichzeitig die komplementäre Ergänzung wahrgenommen, was zu einer  Beeinflussung angrenzender Farbflächen führt.
		\end{itemize}
	\end{frame}
\end{document}
