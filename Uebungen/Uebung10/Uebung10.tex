\documentclass[accentcolor=tud9c,colorbacktitle,inverttitle,landscape,german,presentation,t]{tudbeamer}
\usepackage{amstext}
\usepackage{amsmath}
\usepackage{graphicx}
\usepackage{multicol}
\usepackage{mathtools}
\usepackage{subfigure}
\usepackage[ngerman,english]{babel}
\usepackage[utf8]{inputenc}
\usepackage{colortbl}
\usepackage{adjustbox}

\begin{document}


\title{\"Ubung 10 - Gruppe 142} 
\subtitle{Visual Computing - X3D – 3D in HTML} 

\author[Johannes Beck, Christian Eilers, Robin Menzenbach, Martin Steinborn]{Johannes Beck, Christian Eilers, Robin Menzenbach, Martin Steinborn}


\date{\today}

\begin{titleframe}
\end{titleframe}

\section{Aufgabe 1}
	\begin{frame}
		\frametitle{Aufgabe 1: Rendering}
		%Welche Informationen müssen für das Rendering einer 3D-Szene gegeben sein? 
		%TODO:Nennen Sie zusätzlich zu jedem Punkt ein Beispiel, welches nicht in der Vorlesung vorkam.
		\begin{itemize}
			\item Objekt-Geometrie, z.B.:%TODO
			\item Transformationen, z.B.:%TODO
			\item Materialien, z.B.:%TODO
			\item Kameras, z.B.:%TODO
			\item Lichter, z.B.:%TODO
			\item spezial-Effekte, z.B.:%TODO
		\end{itemize}
	\end{frame}

\section{Aufgabe 2}
\begin{frame}
	\frametitle{Aufgabe 2: X3D-Dokument}
	%Wie ist es möglich, in einem X3D-Dokument bestimmte Elemente mehrfach zu zeichnen, ohne diese mehrfach zu definieren?
	in einem X3D-Dokument ist es möglich bestimmte Elemente mehrfach zu zeichen, ohne sie mehrfach zu definieren, indem man innerhalb einer Gruppierung mehreree Transformationen einfügt, welche alle auf dieselben Objektdaten verweisen.
\end{frame}

\section{Aufgabe 3}
\begin{frame}
	\frametitle{Aufgabe 3: Szenengraph-Standards}
	%TODO:Welche bekannten Szenengraph-Standards gibt es noch neben X3DOM? Nennen Sie mind. zwei und beschreiben Sie diese kurz.
\end{frame}

\section{Aufgabe 4}
\begin{frame}
	\frametitle{Aufgabe 4: Szenengraph – Hands on}
	%TODO:Erstellen Sie aus dem folgenden Bild einen Szenegraphen. Dieser sollte mindestens 6	Gruppierungsknoten enthalten. Enthält Ihr Graph bereits 6 Gruppierungsknoten, stellt aber nicht alle Details des Bildes dar, dann führt das nicht zu Punktabzug. Zeichnen Sie für einen Gruppierungsknoten beispielhaft die Transformations- und Objektknoten
\end{frame}

\section{Aufgabe 5}
\begin{frame}
	\frametitle{Aufgabe 5: Praktikum}
	%TODO:Erstellen Sie gemäß den folgenden Punkten eine X3DOM-Szene. Erstellen sie dazu eine einzige	HTML-Datei für ihr Markup und geben Sie diese separat ab.	Fertigen Sie ein Bild Ihrer Szene an und fügen sie es Ihrer Präsentation hinzu sowie eine Szene,die zunächst nichts bis auf eine Box in einer expliziten Farbe ihrer Wahl, platziert im Ursprung,enthält.	Färben Sie die eben erstellte Box grün.	Fügen Sie drei weitere Boxen hinzu und färben sie diese blau. Platzieren Sie diese Boxen links,rechts und oberhalb der bereits existierenden grünen Box.
\end{frame}
\end{document}
