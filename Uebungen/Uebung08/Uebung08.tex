\documentclass[accentcolor=tud9c,colorbacktitle,inverttitle,landscape,german,presentation,t]{tudbeamer}
\usepackage{amstext}
\usepackage{amsmath}
\usepackage{graphicx}
\usepackage{multicol}
\usepackage{mathtools}
\usepackage{subfigure}
\usepackage[ngerman,english]{babel}
\usepackage[utf8]{inputenc}
\usepackage{colortbl}
\usepackage{adjustbox}

\begin{document}

\title{\"Ubung 8 - Gruppe 142}
\subtitle{Visual Computing - Transformationen \& 2D/3D Ausgabe}

\author[Johannes Beck, Christian Eilers, Robin Menzenbach, Martin Steinborn]{Johannes Beck, Christian Eilers, Robin Menzenbach, Martin Steinborn}


\date{\today}

\begin{titleframe}
\end{titleframe}

\section{Aufgabe 1}
	\begin{frame}
		\frametitle{Aufgabe Aufgabe 1: Projektionen}
		\begin{itemize}
		\item[a)] Bei der perspektivischen Projektion treffen sich die Strahlen im einem Punkt, dem sog. Aufpunkt. Bei der parallelen Projektion, sind die Strahlen, wie der Name schon sagt, parallel zueinander. Dabei wirkt die perspektivische Darstellung natürlicher, jedoch können durch die Projektion Abstände, Längenverhältnisse und Winkel verändert werden. Durch die parallele Projektion, ändern sich Winkel, Längen und damit Abstände nicht, damit bleiben parallele Linien parallel und es können einfacher Längenmessungen durchgeführt werden.
		\item[b)] Anwendungsgebiete gibt es viele für die parallele Projektion, da durch diese die Längen einfach aus der Projektion übernommen werden können. So nutzen Mediziner zum Beispiel diese Art der Projektion, da sie so aus den gescannten Daten ihrer Patienten einfacher ablesen können.
		\item[c)] \begin{itemize}
				\item[A] perspektivische Projektion mit Fluchtpunkt hinter dem Haus
				\item[B] Parallel -> kein Fluchtpunkt
				\item[C] Parallel -> kein Fluchtpunk
				\item[D] perspektivische Projektion mit Fluchtpunkt in der Mitte der Glastür
				\end{itemize}
		\begin{itemize}
	\end{frame}

\section{Aufgabe 2}
\begin{frame}
	\frametitle{Aufgabe 2: Transformationen}
\end{frame}

\section{Aufgabe 3}
\begin{frame}
	\frametitle{Aufgabe 3: Eigenschaften von Rotationsmatrizen} %TODO Formatierung
\end{frame}
\end{document}
