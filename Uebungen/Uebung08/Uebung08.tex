\documentclass[accentcolor=tud9c,colorbacktitle,inverttitle,landscape,german,presentation,t]{tudbeamer}
\usepackage{amstext}
\usepackage{amsmath}
\usepackage{graphicx}
\usepackage{multicol}
\usepackage{mathtools}
\usepackage{subfigure}
\usepackage[ngerman,english]{babel}
\usepackage[utf8]{inputenc}
\usepackage{colortbl}
\usepackage{adjustbox}

\begin{document}

\title{\"Ubung 8 - Gruppe 142}
\subtitle{Visual Computing - Transformationen \& 2D/3D Ausgabe}

\author[Johannes Beck, Christian Eilers, Robin Menzenbach, Martin Steinborn]{Johannes Beck, Christian Eilers, Robin Menzenbach, Martin Steinborn}


\date{\today}

\begin{titleframe}
\end{titleframe}

\section{Aufgabe 1}
	\begin{frame}
		\frametitle{Aufgabe Aufgabe 1: Projektionen}
		\begin{itemize}
		\item[a)] Bei der perspektivischen Projektion treffen sich die Strahlen im einem Punkt, dem sog. Aufpunkt. Bei der parallelen Projektion, sind die Strahlen, wie der Name schon sagt, parallel zueinander. Dabei wirkt die perspektivische Darstellung natürlicher, jedoch können durch die Projektion Abstände, Längenverhältnisse und Winkel verändert werden. Durch die parallele Projektion, ändern sich Winkel, Längen und damit Abstände nicht, damit bleiben parallele Linien parallel und es können einfacher Längenmessungen durchgeführt werden.
		\item[b)] Anwendungsgebiete gibt es viele für die parallele Projektion, da durch diese die Längen einfach aus der Projektion übernommen werden können. So nutzen Mediziner zum Beispiel diese Art der Projektion, da sie so aus den gescannten Daten ihrer Patienten einfacher ablesen können.
		\end{itemize}
	\end{frame}

	\begin{frame}
		\frametitle{Aufgabe Aufgabe 1: Projektionen}
		\begin{itemize}
		\item[c)] 
		\begin{itemize}
			\item[A] perspektivische Projektion mit Fluchtpunkt hinter dem Haus
			\item[B] Parallel -> kein Fluchtpunkt
			\item[C] Parallel -> kein Fluchtpunk
			\item[D] perspektivische Projektion mit Fluchtpunkt in der Mitte der Glastür
		\end{itemize}
		\end{itemize}
	\end{frame}

\section{Aufgabe 2}
\begin{frame}
	\frametitle{Aufgabe 2: Transformationen}
	Die Matrizen der Basistransformationen ergeben sich zu
	$T_{Translation}=\begin{bmatrix}
	1 & 0 & 0 & 0\\
	0 & 1 & 0 & 2\\
	0 & 0 & 1 & 0\\
	0 & 0 & 0 & 1
	\end{bmatrix} $, 
	$T_{Rotation}=\begin{bmatrix}
	0.90 & -0.19 & -0.19 & 0\\
	0.39 & 0.76 & 0.63 & 0\\
	-0.19 & -0.53 & 0.76 & 0\\
	0 & 0 & 0 & 1
	\end{bmatrix} $, 
	$T_{Skalierung}=\begin{bmatrix}
	1 & 0 & 0 & 0\\
	0 & 1 & 0 & 0\\
	0 & 0 & 3 & 0\\
	0 & 0 & 0 & 1
	\end{bmatrix} $, 
	$T_{Scherung}=\begin{bmatrix}
	1 & 0 & 0.5 & 0\\
	2 & 1 & 0.5 & 0\\
	2 & 0 & 1 & 0\\
	0 & 0 & 0 & 1
	\end{bmatrix} $,
	 $T_{Projektion}=\begin{bmatrix}
	 0 & 0 & 0 & 0\\
	 0 & 0 & 0 & 0\\
	 0 & 0 & 1 & 0\\
	 0.5 & 2 & 0 & 1
	 \end{bmatrix} $\\

\end{frame}
\begin{frame}
	\frametitle{Aufgabe 2: Transformationen}
	Damit folgt für die Verkettung der einzuelnen Projektionen\\
	$T_{ges}=T_{Projektion} \cdot T_{Scherung} \cdot T_{Skalierung} \cdot T_{Rotation} \cdot T_{Translation}$\\
	$=\begin{bmatrix}
	 &  &  & \\
	 &  &  & \\
	 &  &  & \\
	 &  &  & 
	\end{bmatrix}$\\
	Damit ergeben sich die Positionen des Würfels zu\\
	{\small 
	$p_1=T_{ges} \cdot \begin{bmatrix} 0\\0\\0\end{bmatrix}= $,
	$p_2=T_{ges} \cdot \begin{bmatrix} 1\\0\\0\end{bmatrix}= $,
	$p_3=T_{ges} \cdot \begin{bmatrix} 1\\1\\0\end{bmatrix}= $,
	$p_4=T_{ges} \cdot \begin{bmatrix} 0\\1\\0\end{bmatrix}= $,\\
	$p_5=T_{ges} \cdot \begin{bmatrix} 0\\0\\1\end{bmatrix}= $,
	$p_6=T_{ges} \cdot \begin{bmatrix} 1\\0\\1\end{bmatrix}= $,
	$p_7=T_{ges} \cdot \begin{bmatrix} 1\\1\\1\end{bmatrix}= $,
	$p_8=T_{ges} \cdot \begin{bmatrix} 0\\1\\1\end{bmatrix}= $.
}
	
\end{frame}

\section{Aufgabe 3}
\begin{frame}
	\frametitle{Aufgabe 3: Eigenschaften von \\ Rotationsmatrizen} 
	\begin{itemize}
		\item[a)] Die Matrix A ist keine Rotationsmatrix, da sie nicht quadratisch ist. somit lässt sie sich auch nicht invertieren. Allerdings hat sie große Ähnlichkeit mit einem Ausschnitt der Rotationsmatrix um die Z-Achse. Allerdings müsste hierfür entweder die Nullzeile entfernt, oder der Einheitsvektor $\begin{bmatrix} 0 & 0 & 1\end{bmatrix}^T$angehängt werden. Anschliesend läge eine Rotationsmatrix vor, deren Inverse sich aus der Transformierten ergibt.%TODO: bitte nochmal anschauen
		\item[b)] $B^{-1}=\begin{bmatrix}
		0 & 1 & 0\\
		cos(25) & 0 & -sin(25)\\
		sin(25) & 0 & cos(25)
		\end{bmatrix} $ \\
		Es handelt sich um die Verkettung einer Drehung um 25 Grad um die y-Achse mit einer Rotation um 180 Grad um die Gerade (1,1,0). Somit ergibt sich die Inverse $B^{-1}$ aus $B^{-1}=B^{T} $
	\end{itemize}
\end{frame}

\begin{frame}
	\frametitle{Aufgabe 3: Eigenschaften von \\ Rotationsmatrizen} 
	\begin{itemize}
		\item[c)] $C^{-1}=\begin{bmatrix}
		cos(\phi)cos(\theta) & sin(\theta) & cos(\phi)\\
		-cos(\phi)sin(\theta) & cos(\theta) & -sin(\phi)sin(\theta) \\
		-sin(\phi) & 0 & cos(\theta) 
		\end{bmatrix} $ \\
		Hier handelt es sich um die Verkettung einer Rotation um die Y-Achse mit einer Rotation um die Z-Achse. Somit handelt es sich um eine Rotation und die Inverse ergibt sich aus der Transformierten der ursprünglichen Matrix $C^{-1}=C^T$
	\end{itemize}
\end{frame}
\end{document}
